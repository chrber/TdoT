%%%%%%%%%%%%%%%%%%%%%%%%%%%%%%%%%%%%%%%%%
% Jacobs Landscape Poster
% LaTeX Template
% Version 1.1 (14/06/14)
%
% Created by:
% Computational Physics and Biophysics Group, Jacobs University
% https://teamwork.jacobs-university.de:8443/confluence/display/CoPandBiG/LaTeX+Poster
%
% Further modified by:
% Nathaniel Johnston (nathaniel@njohnston.ca)
%
% This template has been downloaded from:
% http://www.LaTeXTemplates.com
%
% License:
% CC BY-NC-SA 3.0 (http://creativecommons.org/licenses/by-nc-sa/3.0/)
%
%%%%%%%%%%%%%%%%%%%%%%%%%%%%%%%%%%%%%%%%%

%----------------------------------------------------------------------------------------
%	PACKAGES AND OTHER DOCUMENT CONFIGURATIONS
%----------------------------------------------------------------------------------------

\documentclass[final]{beamer}

\usetheme{confposter} % Use the confposter theme supplied with this template
%\usepackage[scale=1.24, size=a0]{beamerposter} % Use the beamerposter package for laying out the poster
\usepackage[scale=1.0, size=a0, orientation=portrait]{beamerposter} % Use the beamerposter package for laying out the poster
\usepackage{graphicx}  % Required for including images
\usepackage{booktabs} % Top and bottom rules for tables
\usepackage{array}
\usepackage{pst-barcode}

\newcolumntype{C}[1]{>{\centering}m{#1}}
\newcolumntype{L}[1]{>{}m{#1}}
\newcommand{\barcode}[1]{
\url{#1} \begin{pspicture}(0.2in,0.2in) 
\psbarcode[scalex=0.5,scaley=0.5]{#1}{}{qrcode} 
\end{pspicture}
}

%\setbeamercolor{block title}{fg=ngreen,bg=white} % Colors of the block titles
%\setbeamercolor{block body}{fg=black,bg=white} % Colors of the body of blocks
%\setbeamercolor{block alerted title}{fg=white,bg=dblue!70} % Colors of the highlighted block titles
%\setbeamercolor{block alerted body}{fg=black,bg=dblue!10} % Colors of the body of highlighted blocks
% Many more colors are available for use in beamerthemeconfposter.sty

%-----------------------------------------------------------
% Define the column widths and overall poster size
% To set effective sepwid, onecolwid and twocolwid values, first choose how many columns you want and how much separation you want between columns
% In this template, the separation width chosen is 0.024 of the paper width and a 4-column layout
% onecolwid should therefore be (1-(# of columns+1)*sepwid)/# of columns e.g. (1-(4+1)*0.024)/4 = 0.22
% Set twocolwid to be (2*onecolwid)+sepwid = 0.464
% Set threecolwid to be (3*onecolwid)+2*sepwid = 0.708

\newlength{\sepwid}
\newlength{\onecolwid}
\newlength{\twocolwid}
\newlength{\threecolwid}
\setlength{\sepwid}{0.024\textwidth} % Separation width (white space) between columns
\setlength{\onecolwid}{0.22\textwidth} % Width of one column
\setlength{\twocolwid}{0.464\textwidth} % Width of two columns
\setlength{\threecolwid}{0.708\textwidth} % Width of three columns
\setlength{\topmargin}{-0.5in} % Reduce the top margin size
%-----------------------------------------------------------


%----------------------------------------------------------------------------------------
%	TITLE SECTION
%----------------------------------------------------------------------------------------

\title{\includegraphics[]{pics/Sicherheit_header.pdf}} % Poster title

%\author{} % Author(s)

%\institute{Department and University Name} % Institution(s)

%----------------------------------------------------------------------------------------

\begin{document}

\addtobeamertemplate{block end}{}{\vspace*{2ex}} % White space under blocks
\addtobeamertemplate{block alerted end}{}{\vspace*{2ex}} % White space under highlighted (alert) blocks

\setlength{\belowcaptionskip}{2ex} % White space under figures
\setlength\belowdisplayshortskip{2ex} % White space under equations

\begin{frame}[t] % The whole poster is enclosed in one beamer frame

\begin{columns}[b]

\begin{column}{0.95\textwidth}

\begin{alertblock}{\large{DESY als Angriffsziel? - Ressourcen, die geschützt werden müssen.}}
\includegraphics[width=2.0\twocolwid]{pics/WhatToProtect.pdf}
\end{alertblock}
\end{column}
\end{columns}

\begin{columns}[t] % The whole poster consists of three major columns, the second of which is split into two columns twice - the [t] option aligns each column's content to the top

\begin{column}{0.95\textwidth} % The first column

%----------------------------------------------------------------------------------------
%	INTRODUCTION
%----------------------------------------------------------------------------------------

\begin{block}{IT-Sicherheit in DESY}

Das Internet startete als Mittel zum Austausch von Informationen
unter Wissenschaftlern. Natürlich benutzen auch heute Wissenschaftler das Internet und sind den gleichen
Gefahren ausgesetzt, denen auch Sie ausgesetzt sind. DESY hat auch die Aufgabe, die angeschafften
Ressourcen und deren Nutzer vor Angriffen Dritter zu schützen. Dies ist die Aufgabe der IT-Sicherheit. 
Selbstverständlich sind auch die DESY-Nutzer dazu angehalten, sich selbst und damit DESY zu schützen, indem sie ihre Informationssysteme auf dem neusten Stand halten. Die Ma{\ss}nahmen, die wir bei DESY treffen, um uns zu schützen, sind die gleichen, die auch Sie nutzen können. Dieser Stand gibt Einblick in verschiedene Aspekte der DESY IT-Sicherheit.

\end{block}

\begin{block}{\Huge {An diesem Stand finden Sie:}}

\Huge{
\begin{center}
\begin{itemize}
    \centering 
    \item [] Warum DESY sich schützen muss?
    \item [] Wie DESY sich schützt?
    \item [] Wie vor Phishing schützen?
    \item [] Welche Programme sind sicher/unsicher?
\end{itemize}
\end{center}
}

\end{block}

\end{column}

\end{columns} % End of all the columns in the poster


\end{frame} % End of the enclosing frame

\end{document}
