%%%%%%%%%%%%%%%%%%%%%%%%%%%%%%%%%%%%%%%%%
% Jacobs Landscape Poster
% LaTeX Template
% Version 1.1 (14/06/14)
%
% Created by:
% Computational Physics and Biophysics Group, Jacobs University
% https://teamwork.jacobs-university.de:8443/confluence/display/CoPandBiG/LaTeX+Poster
%
% Further modified by:
% Nathaniel Johnston (nathaniel@njohnston.ca)
%
% This template has been downloaded from:
% http://www.LaTeXTemplates.com
%
% License:
% CC BY-NC-SA 3.0 (http://creativecommons.org/licenses/by-nc-sa/3.0/)
%
%%%%%%%%%%%%%%%%%%%%%%%%%%%%%%%%%%%%%%%%%

%----------------------------------------------------------------------------------------
%	PACKAGES AND OTHER DOCUMENT CONFIGURATIONS
%----------------------------------------------------------------------------------------

\documentclass[final]{beamer}

\usetheme{confposter} % Use the confposter theme supplied with this template
%\usepackage[scale=1.24, size=a0]{beamerposter} % Use the beamerposter package for laying out the poster
\usepackage[scale=1.0, size=a0, orientation=portrait]{beamerposter} % Use the beamerposter package for laying out the poster
\usepackage{graphicx}  % Required for including images
\usepackage{booktabs} % Top and bottom rules for tables
\usepackage{array}
\usepackage{pst-barcode}

\newcolumntype{C}[1]{>{\centering}m{#1}}
\newcolumntype{L}[1]{>{}m{#1}}
\newcommand{\barcode}[1]{
\url{#1} \begin{pspicture}(0.2in,0.2in) 
\psbarcode[scalex=0.5,scaley=0.5]{#1}{}{qrcode} 
\end{pspicture}
}

%\setbeamercolor{block title}{fg=ngreen,bg=white} % Colors of the block titles
%\setbeamercolor{block body}{fg=black,bg=white} % Colors of the body of blocks
%\setbeamercolor{block alerted title}{fg=white,bg=dblue!70} % Colors of the highlighted block titles
%\setbeamercolor{block alerted body}{fg=black,bg=dblue!10} % Colors of the body of highlighted blocks
% Many more colors are available for use in beamerthemeconfposter.sty

%-----------------------------------------------------------
% Define the column widths and overall poster size
% To set effective sepwid, onecolwid and twocolwid values, first choose how many columns you want and how much separation you want between columns
% In this template, the separation width chosen is 0.024 of the paper width and a 4-column layout
% onecolwid should therefore be (1-(# of columns+1)*sepwid)/# of columns e.g. (1-(4+1)*0.024)/4 = 0.22
% Set twocolwid to be (2*onecolwid)+sepwid = 0.464
% Set threecolwid to be (3*onecolwid)+2*sepwid = 0.708

\newlength{\sepwid}
\newlength{\onecolwid}
\newlength{\twocolwid}
\newlength{\threecolwid}
\setlength{\sepwid}{0.024\textwidth} % Separation width (white space) between columns
\setlength{\onecolwid}{0.22\textwidth} % Width of one column
\setlength{\twocolwid}{0.464\textwidth} % Width of two columns
\setlength{\threecolwid}{0.708\textwidth} % Width of three columns
\setlength{\topmargin}{-0.5in} % Reduce the top margin size
%-----------------------------------------------------------


%----------------------------------------------------------------------------------------
%	TITLE SECTION
%----------------------------------------------------------------------------------------

\title{Netzwerksicherheit am DESY} % Poster title

\author{Carsten Porthun, Christian Bernardt, Karsten Schwank} % Author(s)

%\institute{Department and University Name} % Institution(s)

%----------------------------------------------------------------------------------------

\begin{document}

\addtobeamertemplate{block end}{}{\vspace*{2ex}} % White space under blocks
\addtobeamertemplate{block alerted end}{}{\vspace*{2ex}} % White space under highlighted (alert) blocks

\setlength{\belowcaptionskip}{2ex} % White space under figures
\setlength\belowdisplayshortskip{2ex} % White space under equations

\begin{frame}[t] % The whole poster is enclosed in one beamer frame

\begin{columns}[b]

\begin{column}{0.95\textwidth}

\begin{alertblock}{DESY als Angriffsziel?}
\includegraphics[width=2.0\twocolwid]{pics/WhatToProtect.pdf}
\end{alertblock}
\end{column}
\end{columns}

\begin{columns}[t, totalwidth=\textwidth] % The whole poster consists of three major columns, the second of which is split into two columns twice - the [t] option aligns each column's content to the top

\begin{column}{\sepwid}\end{column} % Empty spacer column

\begin{column}{0.9\twocolwid} % The first column

%----------------------------------------------------------------------------------------
%	INTRODUCTION
%----------------------------------------------------------------------------------------

\begin{block}{Netzwerksicherheit in DESY}

Das Internet startete als Mittel zum Austausch von Informationen
unter Wissenschaftlern. Natürlich benutzen auch heute Wissenschaftler das Internet und sind den gleichen
Gefahren ausgesetzt, denen auch Sie ausgesetzt sind. DESY hat natürlich auch die Aufgabe, die angeschafften
Resourcen und deren Nutzer vor Angriffen Dritter zu schützen. Dies ist die Aufgabe der Netzwerksicherheit. Natürlich ist auch jeder Wissenschaftler dazu angehalten, sich selbst und damit DESY zu schützen, indem Sie sichere Programme nutzen und ihre Informationssysteme auf dem neusten Stand halten. Die Maßnahmen, die Wissenschaftler treffen, um sich zu schützen sind die gleichen, die auch Sie nutzen können, um besser geschützt zu sein. Hundertprozentiger Schutz ist nicht erreichbar, aber einiges kann man schon machen. Eine wichtige Maßnahme ist, sichere Programme zu verwenden, was auch für Sie wichtig ist.

\end{block}

\begin{block}{Welche Programme sind sicher?}
Oft weisen selbst als sicher vermarktete Programme massive Sicherheitslücken auf. \barcode{https://www.eff.org/secure-messaging-scorecard} hat hier einen sehr schönen Vergleich dargestellt, der zeigt was einige der beliebtesten Programme an Lücken aufweisen. Tabelle \ref{comparePrograms} zeigt eine Auswahl aus diesem Vergleich. In der Darstellung befinden sich aber auch weniger bekannte Programme, die mehr Sicherheit versprechen und vom interessierten Leser einmal getestet werden könnten.
\linebreak
\end{block}

\end{column} % End of the first column

\begin{column}{\sepwid}\end{column} % Empty spacer column

\begin{column}{1.2\twocolwid} % Begin a column which is two columns wide (column 2)

\begin{tabular}{ L{8cm} | C{4.5cm} | C{4.5cm} | C{4.5cm} | C{4.5cm} | C{4.5cm} | C{5cm} | C{4.5cm} | }
\label{comparePrograms}

Prog. & Ver\-schlüs\-sel\-te Über\-tra\-gung? & Un\-sicht\-bar beim Pro\-vider? & Iden\-tität be\-kannt? & Frühere Daten sicher, wenn Schlüssel ge\-stohlen? & Sourcen offen? & Programm Design dokumentiert? & Gab es einen zeitnahen code Audit? \tabularnewline
\hline
AIM & \includegraphics[scale=0.5]{pics/haken.png} & \includegraphics[scale=0.5]{pics/nohaken.png} & \includegraphics[scale=0.5]{pics/nohaken.png} & \includegraphics[scale=0.5]{pics/nohaken.png} & \includegraphics[scale=0.5]{pics/nohaken.png} & \includegraphics[scale=0.5]{pics/nohaken.png} & \includegraphics[scale=0.5]{pics/nohaken.png} \tabularnewline[0.5ex]
Facebook chat & \includegraphics[scale=0.5]{pics/haken.png} & \includegraphics[scale=0.5]{pics/nohaken.png} & \includegraphics[scale=0.5]{pics/nohaken.png} & \includegraphics[scale=0.5]{pics/nohaken.png} & \includegraphics[scale=0.5]{pics/nohaken.png} & \includegraphics[scale=0.5]{pics/nohaken.png} & \includegraphics[scale=0.5]{pics/haken.png} \tabularnewline
Skype & \includegraphics[scale=0.5]{pics/haken.png} & \includegraphics[scale=0.5]{pics/nohaken.png} & \includegraphics[scale=0.5]{pics/nohaken.png} & \includegraphics[scale=0.5]{pics/nohaken.png} & \includegraphics[scale=0.5]{pics/nohaken.png} & \includegraphics[scale=0.5]{pics/nohaken.png} & \includegraphics[scale=0.5]{pics/nohaken.png} \tabularnewline
FaceTime & \includegraphics[scale=0.5]{pics/haken.png} & \includegraphics[scale=0.5]{pics/haken.png} & \includegraphics[scale=0.5]{pics/nohaken.png} & \includegraphics[scale=0.5]{pics/haken.png} & \includegraphics[scale=0.5]{pics/nohaken.png} & \includegraphics[scale=0.5]{pics/haken.png} & \includegraphics[scale=0.5]{pics/haken.png} \tabularnewline
WhatsApp & \includegraphics[scale=0.5]{pics/haken.png} & \includegraphics[scale=0.5]{pics/nohaken.png} & \includegraphics[scale=0.5]{pics/nohaken.png} & \includegraphics[scale=0.5]{pics/nohaken.png} & \includegraphics[scale=0.5]{pics/nohaken.png} & \includegraphics[scale=0.5]{pics/nohaken.png} & \includegraphics[scale=0.5]{pics/haken.png} \tabularnewline
Google Hangout & x & y & y & y & y & y & x \tabularnewline
iMessage & x & x & y & x & y & x & x \tabularnewline
Jitsi+Ostel  & x & x & x & x & x & x & y \tabularnewline
Mailvelope   & x & x & x & y & x & x & x \tabularnewline
Adium+OTR & x & x & x & x & x & x & y \tabularnewline
Pidgin+OTR & x & x & x & x & x & x & x \tabularnewline
GPGTools & x & x & x & y & x & x & y \tabularnewline
Retroshare & x & x & x & x & x & x & y \tabularnewline
Signal / Redphone & x & x & x & x & x & x & x \tabularnewline
Silent Phone & x & x & x & x & x & x & x \tabularnewline
Silent Text & x & x & x & x & x & x & x \tabularnewline
Subrosa & x & x & x & y & x & x & x \tabularnewline
SureSpot & x & x & x & y & x & x & y \tabularnewline
Telegram (secret chat) & x & x & x & x & x & x & x \tabularnewline
TextSecure & x & x & x & x & x & x & x \tabularnewline
\hline
\end{tabular}

\end{column} % End of the second column

\begin{column}{\sepwid}\end{column} % Empty spacer column

\end{columns} % End of all the columns in the poster


\end{frame} % End of the enclosing frame

\end{document}
