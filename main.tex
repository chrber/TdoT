%%%%%%%%%%%%%%%%%%%%%%%%%%%%%%%%%%%%%%%%%
% Jacobs Landscape Poster
% LaTeX Template
% Version 1.1 (14/06/14)
%
% Created by:
% Computational Physics and Biophysics Group, Jacobs University
% https://teamwork.jacobs-university.de:8443/confluence/display/CoPandBiG/LaTeX+Poster
%
% Further modified by:
% Nathaniel Johnston (nathaniel@njohnston.ca)
%
% This template has been downloaded from:
% http://www.LaTeXTemplates.com
%
% License:
% CC BY-NC-SA 3.0 (http://creativecommons.org/licenses/by-nc-sa/3.0/)
%
%%%%%%%%%%%%%%%%%%%%%%%%%%%%%%%%%%%%%%%%%

%----------------------------------------------------------------------------------------
%	PACKAGES AND OTHER DOCUMENT CONFIGURATIONS
%----------------------------------------------------------------------------------------

\documentclass[final]{beamer}

%\usepackage[scale=1.24, size=a0]{beamerposter} % Use the beamerposter package for laying out the poster
\usepackage[scale=1.24, size=a0, orientation=portrait]{beamerposter} % Use the beamerposter package for laying out the poster



\usetheme{confposter} % Use the confposter theme supplied with this template

%\setbeamercolor{block title}{fg=ngreen,bg=white} % Colors of the block titles
%\setbeamercolor{block body}{fg=black,bg=white} % Colors of the body of blocks
%\setbeamercolor{block alerted title}{fg=white,bg=dblue!70} % Colors of the highlighted block titles
%\setbeamercolor{block alerted body}{fg=black,bg=dblue!10} % Colors of the body of highlighted blocks
% Many more colors are available for use in beamerthemeconfposter.sty

%-----------------------------------------------------------
% Define the column widths and overall poster size
% To set effective sepwid, onecolwid and twocolwid values, first choose how many columns you want and how much separation you want between columns
% In this template, the separation width chosen is 0.024 of the paper width and a 4-column layout
% onecolwid should therefore be (1-(# of columns+1)*sepwid)/# of columns e.g. (1-(4+1)*0.024)/4 = 0.22
% Set twocolwid to be (2*onecolwid)+sepwid = 0.464
% Set threecolwid to be (3*onecolwid)+2*sepwid = 0.708

\newlength{\sepwid}
\newlength{\onecolwid}
\newlength{\twocolwid}
\newlength{\threecolwid}
\setlength{\sepwid}{0.024\textwidth} % Separation width (white space) between columns
\setlength{\onecolwid}{0.22\textwidth} % Width of one column
\setlength{\twocolwid}{0.464\textwidth} % Width of two columns
\setlength{\threecolwid}{0.708\textwidth} % Width of three columns
\setlength{\topmargin}{-0.5in} % Reduce the top margin size
%-----------------------------------------------------------

\usepackage{graphicx}  % Required for including images

\usepackage{booktabs} % Top and bottom rules for tables

%----------------------------------------------------------------------------------------
%	TITLE SECTION
%----------------------------------------------------------------------------------------

\title{Netzwerksicherheit am DESY} % Poster title

\author{Carsten Porthun, Christian Bernardt, Karsten Schwank} % Author(s)

%\institute{Department and University Name} % Institution(s)

%----------------------------------------------------------------------------------------

\begin{document}

\addtobeamertemplate{block end}{}{\vspace*{2ex}} % White space under blocks
\addtobeamertemplate{block alerted end}{}{\vspace*{2ex}} % White space under highlighted (alert) blocks

\setlength{\belowcaptionskip}{2ex} % White space under figures
\setlength\belowdisplayshortskip{2ex} % White space under equations

\begin{frame}[t] % The whole poster is enclosed in one beamer frame

\begin{columns}[t, totalwidth=\textwidth] % The whole poster consists of three major columns, the second of which is split into two columns twice - the [t] option aligns each column's content to the top

\begin{column}{\sepwid}\end{column} % Empty spacer column

\begin{column}{\twocolwid} % The first column

%----------------------------------------------------------------------------------------
%	INTRODUCTION
%----------------------------------------------------------------------------------------

\begin{block}{Einleitung}

Das Internet ist ein offener Zusammenschluss aus Milliarden von unterschiedlichsten
Teilnehmern. Während der ursprüngliche Zweck des Internets der Austausch von Informationen
unter Wissenschaftlern war, hat heute nahezu jede Organisation eine Internetpräsens und
es tangiert nahezu all unsere Lebensbereiche. Diese Tatsache macht das Internet leider auch
zu einem Werkzeug von unterschiedlichen Angreifern, die versuchen illegal Gewinn aus dieser
Tatsache zu ziehen.
Die Angreifer nutzen unterschiedlichste Taktiken und je nach Ziel des Angreifers kann jeder
Internetteilnehmer von diesen Angriffen betroffen sein:

\begin{itemize}
 \item Öffentliche Server sind häufig mit hoher Bandbreite an das Internet angebunden; das 
 macht diese Systeme als Ziel interessant, da sich darüber schnell Daten verschicken lassen.
 \item Viele Organisationen haben auch interne Computer an das Internet angeschlossen, um 
 zum Beispiel unterschiedliche Filialen zu vernetzen. Diese System enthalten oft geheime Daten, 
 wie Kundenpasswörter oder interne Dokumente.
 \item Private Benutzer haben häufig schwach gesicherte Computer, so dass sie leicht massenhaft
 automatisiert kompromittiert werden können. Die Angreifer können dann Computer weiter ausspionieren, 
 um zum Beispiel Passwörter zu erlangen oder um den Computer für Angriffe auf weitere Systeme 
 zu missbrauchen.
\end{itemize}

\end{block}

%----------------------------------------------------------------------------------------
%	OBJECTIVES
%----------------------------------------------------------------------------------------

\begin{column}{\sepwid}\end{column} % Empty spacer column	

\begin{alertblock}{DESY als Angriffsziel?}
\small{
DESY ist gleich aus mehreren Gründen für Angreifer ein attraktives Ziel. 

\begin{itemize}
\item Bandbreite: DESY hat eine sehr schnelle Anbindung an das Internet 
(20 Gbit/s). Diese Bandbreite könnte für weitere Angriffe missbraucht 
werden, zum Beispiel zum massenhaften Versenden von Spam.

\item Rechenleistung: Die Analyse von wissenschaftlichen Daten erfordert 
viel Rechenleistung. Diese Rechenleistung könnte von Angreifern missbraucht werden,
um Passwörter zu entschlüsseln, Bilderkennung zu betreiben, Medien yu streamen oder andere rechenintensive und damit kostspielige Aktionen auszuführen.

\item  Speicherplatz: Die Massenspeicher am DESY haben eine GröSSe von über 
15 Petabyte. Dieser Speicherplatz hat einen hohen Wert und kann zusammen 
mit der Bandbreite zum unerlaubten Verbreiten Von Dateien (z.B. urheberrechtlich 
geschützte Werke) missbraucht werden.

\end{itemize}
}
\end{alertblock}

%------------------------------------------------

\begin{block}{Wie wird DESY angegriffen?}

Angreifer bedienen sich unterschiedlichster Methoden, um fremde Systeme zu 
kompromittieren. Normale Angriffe lassen sich in zwei Kategorien aufteilen:

\begin{itemize}
 \item Ausnutzen von Software oder Konfigurationsschwachstellen.
 \item Täuschung einzelner Mitarbeiter, um deren Zugriffsrechte zu missbrauchen.
\end{itemize}

Mächtigere Angreifer mit Zugriff auf Teile der Internet-Infrastruktur 
(z.B. Geheimdienste) haben neben diesen Methoden noch die Möglichkeit 
Netzwerkverbindungen abzuhören und zu manipulieren.

\end{block}

%----------------------------------------------------------------------------------------

\end{column} % End of the first column

\begin{column}{\sepwid}\end{column} % Empty spacer column

\begin{column}{\twocolwid} % Begin a column which is two columns wide (column 2)

\begin{block}{Software- \& Konfigurationsschwachstellen}
 Die hohe Komplexität in heutigen Softwaresystemen und Infrastrukturen führt dazu dass
 Sicherheitslücken entstehen können, die von Angreifern ausgenutzt werden können. 
 Gerade weit verbreitete Software, wie Betriebssysteme, Web- und Datenbankserver werden 
 von Angreifern auf Schwachstellen analysiert, die dann den Angriff auf viele Ziele ermöglicht, 
 die die gleiche Software einsetzen. Leider haben verschiedene Geheimdienste
 aktiv darauf hingewirkt Sicherheitsmechanismen zu 		schwächen. Diese Schwachstellen können
 natürlich auch von nicht legitimierten Angreifern ausgenutzt werden.
 \par
 Konfigurationsfehler, z.B. Standardpasswörter oder veraltete Softwarepakete können
 Angreifern erlauben in interne Netzwerke einzubrechen und von dort weitere Systeme 
 unter Kontrolle zu bringen.
\end{block}

\begin{block}{Angriffe auf Benutzer}
 Der weitaus grössere Anteil der Angriffe nutzt die Achtlosigkeit der Benutzer aus. 
 Bei gezielten Angriffen wird z.B. versucht sich über Fishing-eMails deren Zugriffsrechte anzueignen. 
 Diese eMails werden zum Teil speziell für die Mitarbeiter einer Organisation zugeschnitten 
 und gleichen diesen unter Umständen selbst in feinen Details. Um solche eMails dennoch 
 entlarven zu können, müssen die Benutzer geschult werden.
 \par
 Aber auch ungezielt können die Benutzer ungewollt zu Schwachstellen in der Sicherheit werden:
 Webbrowser und deren Plugins (z.B. das Flash-Plugin) haben in der Vergangenheit häufig
 Fehler enthalten, die es Angreifern erlaubt, Schadcode auf den Computern der Anwender zu 
 installieren.
 \par
 Verlorene oder gestohlene Geräte, z.B. SmartPhones speichern Passwörter und können sensible 
 Daten enthalten. Angreifer können diese Daten auslesen und für einen Angriff nutzen.
\end{block}

\end{column} % End of the second column

\begin{column}{\sepwid}\end{column} % Empty spacer column

\end{columns} % End of all the columns in the poster
\begin{columns}

\begin{column}{0.95\textwidth}

\begin{alertblock}{Schutz für Benutzer}
\small{
Um im Internet nicht schutzlos unterwegs zu sein, sollten Benutzer einige Verhaltensweisen
beherzigen:

\begin{itemize}
 \item Nutzen Sie wenn möglich verschlüsselte Verbindungen (https) zu Webservern, insbesondere
 wenn diese sensible Daten abfragen.\par
 Tools: 
 HTTPS Everywhere: \includegraphics[height=0.7\baselineskip]{httpseverywhere}

 \item Bei eMails unbekannter Herkunft oder eMails, die Aufforderungen enthalten, Zugangsdaten
 preiszugeben ist besondere Vorsicht geboten. Überprüfen Sie den Absender und eventuell enthaltene
 Links. Öffnen Sie Anhänge nur, wenn die Herkunft der eMail zweifelsfrei geklärt ist!
 
 \item Halten Sie Ihr Betriebssystem und Programme - insbesondere Webbrowser und deren Plugins - 
 immer aktuell!
 
 \item Informieren Sie sich über aktuelle Sicherheitslücken. 
 \includegraphics[height=0.7\baselineskip]{sicherimnetz}
 
 \item Nutzen Sie Sicherheitsplugins, die verhindern dass Webseiten unkontrollierbar weitere
 Ressourcen nachladen! \par
 Tools: 
 NoScript: \includegraphics[height=0.7\baselineskip]{noscript}, 
 $\mu{}$Matrix \includegraphics[height=0.7\baselineskip]{umatrix}

 \item Vermeiden Sie unnötige Preisgabe Ihrer Daten! Schon durch die Verknüpfung Ihrer 
 Verbindungsdaten können detaillierte Profile erstellt werden, die es erlauben Sie gezielt zu 
 manipulieren.
 Tools:
 Tracking kontrollieren: \includegraphics[height=0.7\baselineskip]{ghostery}
 Werbenetzwerke erkennen: \includegraphics[height=0.7\baselineskip]{lightbeam}
 
 \item Verschlüsseln Sie alle sensiblen Daten! Emails, Chatprotokolle und andere Dateien können
 von Angreifern leicht ausgelesen werden, wenn sie nicht verschlüsselt sind. 
 Tools: 
 eMail-Verschlüsselung mit Enigmail: \includegraphics[height=0.7\baselineskip]{enigmail},
 Vergleich unterschiedlicher Messenger: \includegraphics[height=0.7\baselineskip]{effmessenger}
 
 \item Geben Sie keine Zugangsdaten in unverschlüsselten WLAN-Netzwerken ein. Mobilgeräte tun
 dies unter Umständen ohne Ihr Zutun, sobald Sie sich verbinden. Benutzen Sie ein VPN 
 \includegraphics[height=0.7\baselineskip]{vpn}, wenn sie öffentliche Hotspots nutzen möchten.
 
 \item Viele Produkte haben freie Alternativen, die in vielen Fällen eine bessere Sicherheit
 und besseren Datenschutz für die Anwender bieten. Eine Liste von interessanten Programmen finden
 Sie unter hier: \includegraphics[height=0.7\baselineskip]{stopprism}, \includegraphics[height=0.7\baselineskip]{prismbreak}
\end{itemize}
}
\end{alertblock}

\end{column}

\end{columns}


\end{frame} % End of the enclosing frame

\end{document}
